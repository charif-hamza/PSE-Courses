\documentclass[10pt,landscape,a4paper]{article}
\usepackage[utf8]{inputenc}
\usepackage[margin=0.5in]{geometry}
\usepackage{amsmath,amssymb,amsfonts}
\usepackage{multicol}
\usepackage{enumitem}
\usepackage{xcolor}
\usepackage{tcolorbox}
\usepackage{fancyhdr}
\usepackage{hyperref}

\pagestyle{fancy}
\fancyhf{}
\lhead{PSE Tier 1: Calculus Cheat Sheet}
\rhead{Page \thepage}
\renewcommand{\headrulewidth}{0.4pt}
\renewcommand{\footrulewidth}{0.4pt}

\setlength{\parindent}{0pt}
\setlength{\parskip}{0.5em}
\setlist{nosep}

\tcbuselibrary{skins,breakable}
\newtcolorbox{conceptbox}[1]{
  colback=blue!5!white,
  colframe=blue!75!black,
  fonttitle=\bfseries,
  title=#1,
  breakable
}

\newtcolorbox{formulabox}{
  colback=green!5!white,
  colframe=green!65!black,
  breakable
}

\newtcolorbox{notebox}{
  colback=yellow!5!white,
  colframe=orange!75!black,
  breakable
}

\begin{document}

\begin{center}
{\LARGE \textbf{Calculus for Process Systems Engineering}} \\
\vspace{0.3em}
{\large Essential Formulas, Concepts, and PSE Applications}
\end{center}

\begin{multicols}{3}

\begin{conceptbox}{Derivatives - Definition}
\textbf{First Principles:}
$$f'(x) = \lim_{h \to 0} \frac{f(x+h) - f(x)}{h}$$

\textbf{Physical Meaning:} Instantaneous rate of change

\textbf{PSE Context:}
\begin{itemize}[leftmargin=*]
\item Reaction rate: $\frac{dC_A}{dt}$
\item Temperature gradient: $\frac{dT}{dx}$
\item Velocity: $\frac{dx}{dt}$
\end{itemize}
\end{conceptbox}

\begin{formulabox}
\textbf{Differentiation Rules:}
\begin{align*}
\frac{d}{dx}[x^n] &= nx^{n-1} \\
\frac{d}{dx}[e^x] &= e^x \\
\frac{d}{dx}[\ln x] &= \frac{1}{x} \\
\frac{d}{dx}[\sin x] &= \cos x \\
\frac{d}{dx}[\cos x] &= -\sin x
\end{align*}

\textbf{Product Rule:}
$$(fg)' = f'g + fg'$$

\textbf{Quotient Rule:}
$$\left(\frac{f}{g}\right)' = \frac{f'g - fg'}{g^2}$$

\textbf{Chain Rule:}
$$\frac{d}{dx}[f(g(x))] = f'(g(x)) \cdot g'(x)$$
\end{formulabox}

\begin{conceptbox}{PSE Application: Arrhenius Equation}
Rate constant temperature dependence:
$$k(T) = A e^{-E_a/(RT)}$$

Temperature sensitivity:
$$\frac{dk}{dT} = k \cdot \frac{E_a}{RT^2}$$

\textbf{Key Insight:} Reactions more sensitive at lower T!
\end{conceptbox}

\begin{formulabox}
\textbf{Integration - Fundamental Theorem:}
$$\int_a^b f'(x)dx = f(b) - f(a)$$

\textbf{Basic Integrals:}
\begin{align*}
\int x^n dx &= \frac{x^{n+1}}{n+1} + C \quad (n \neq -1) \\
\int \frac{1}{x} dx &= \ln|x| + C \\
\int e^x dx &= e^x + C \\
\int \sin x dx &= -\cos x + C \\
\int \cos x dx &= \sin x + C
\end{align*}
\end{formulabox}

\begin{conceptbox}{Integration Techniques}
\textbf{Substitution (u-sub):}
$$\int f(g(x))g'(x)dx = \int f(u)du$$
where $u = g(x)$, $du = g'(x)dx$

\textbf{Integration by Parts:}
$$\int u \, dv = uv - \int v \, du$$

\textbf{LIATE Rule for choosing $u$:}
\begin{enumerate}
\item \textbf{L}ogarithmic
\item \textbf{I}nverse trig
\item \textbf{A}lgebraic
\item \textbf{T}rigonometric
\item \textbf{E}xponential
\end{enumerate}

\textbf{Partial Fractions:}
For $\frac{P(x)}{Q(x)}$, factor $Q(x)$ and decompose:
$$\frac{1}{(x-a)(x-b)} = \frac{A}{x-a} + \frac{B}{x-b}$$
\end{conceptbox}

\begin{notebox}
\textbf{Common Pitfall:} Don't forget $+C$ in indefinite integrals!

\textbf{Check Your Work:} Differentiate your answer to verify.
\end{notebox}

\columnbreak

\begin{conceptbox}{Partial Derivatives}
For $f(x,y)$, partial derivative w.r.t. $x$ holds $y$ constant:
$$\frac{\partial f}{\partial x} = \lim_{h \to 0} \frac{f(x+h, y) - f(x, y)}{h}$$

\textbf{Notation:} $f_x$, $\partial f/\partial x$, $\partial_x f$

\textbf{Mixed Partials (if continuous):}
$$\frac{\partial^2 f}{\partial x \partial y} = \frac{\partial^2 f}{\partial y \partial x}$$
\end{conceptbox}

\begin{formulabox}
\textbf{Chain Rule (Multivariable):}

For $z = f(x,y)$ where $x = x(t)$, $y = y(t)$:
$$\frac{dz}{dt} = \frac{\partial f}{\partial x}\frac{dx}{dt} + \frac{\partial f}{\partial y}\frac{dy}{dt}$$

\textbf{Total Differential:}
$$df = \frac{\partial f}{\partial x}dx + \frac{\partial f}{\partial y}dy$$

\textbf{PSE Example - Ideal Gas:}
$$P = \frac{nRT}{V}$$
$$\left(\frac{\partial P}{\partial T}\right)_V = \frac{nR}{V}, \quad \left(\frac{\partial P}{\partial V}\right)_T = -\frac{nRT}{V^2}$$
\end{formulabox}

\begin{conceptbox}{Optimization}
\textbf{Single Variable:}
\begin{enumerate}
\item Find critical points: $f'(x) = 0$
\item Test: $f''(x) > 0$ (min), $f''(x) < 0$ (max)
\end{enumerate}

\textbf{Multivariable:}
Critical points satisfy:
$$\frac{\partial f}{\partial x} = 0, \quad \frac{\partial f}{\partial y} = 0$$

Use Hessian matrix for classification.
\end{conceptbox}

\begin{formulabox}
\textbf{Taylor Series:}
$$f(x) = \sum_{n=0}^\infty \frac{f^{(n)}(a)}{n!}(x-a)^n$$

\textbf{Linear Approximation:}
$$f(x) \approx f(a) + f'(a)(x-a)$$

\textbf{PSE Use:} Linearizing nonlinear models for control
\end{formulabox}

\begin{notebox}
\textbf{Thermodynamics Notation:}

Always specify what's held constant!
$$\left(\frac{\partial P}{\partial T}\right)_V \neq \left(\frac{\partial P}{\partial T}\right)_S$$
Subscript indicates constant variable.
\end{notebox}

\columnbreak

\begin{conceptbox}{Multiple Integrals}
\textbf{Double Integral:}
$$\iint_R f(x,y) \, dA = \int_a^b \int_{g_1(x)}^{g_2(x)} f(x,y) \, dy \, dx$$

\textbf{Triple Integral:}
$$\iiint_V f(x,y,z) \, dV$$

\textbf{Applications:}
\begin{itemize}
\item Volume under surface
\item Total mass: $\iiint \rho \, dV$
\item Average value: $\bar{f} = \frac{1}{V}\iiint f \, dV$
\end{itemize}
\end{conceptbox}

\begin{formulabox}
\textbf{Coordinate Systems:}

\textbf{Cylindrical $(r, \theta, z)$:}
\begin{align*}
x &= r\cos\theta, \quad y = r\sin\theta, \quad z = z \\
dV &= r \, dr \, d\theta \, dz
\end{align*}

\textbf{Spherical $(\rho, \theta, \phi)$:}
\begin{align*}
x &= \rho\sin\phi\cos\theta \\
y &= \rho\sin\phi\sin\theta \\
z &= \rho\cos\phi \\
dV &= \rho^2 \sin\phi \, d\rho \, d\theta \, d\phi
\end{align*}
\end{formulabox}

\begin{notebox}
\textbf{Critical:} Don't forget Jacobian factors!
\begin{itemize}
\item Cylindrical: factor of $r$
\item Spherical: factor of $\rho^2 \sin\phi$
\end{itemize}

\textbf{Strategy:} Choose coordinates matching problem symmetry.
\end{notebox}

\begin{conceptbox}{Vector Calculus - Gradient}
\textbf{Gradient} of scalar field $f$:
$$\nabla f = \frac{\partial f}{\partial x}\hat{i} + \frac{\partial f}{\partial y}\hat{j} + \frac{\partial f}{\partial z}\hat{k}$$

\textbf{Properties:}
\begin{itemize}
\item Points in direction of steepest increase
\item Perpendicular to level surfaces
\item Magnitude = max rate of change
\end{itemize}

\textbf{PSE: Fourier's Law}
$$\vec{q} = -k \nabla T$$
Heat flux proportional to temperature gradient
\end{conceptbox}

\begin{formulabox}
\textbf{Divergence:}
$$\nabla \cdot \vec{F} = \frac{\partial F_x}{\partial x} + \frac{\partial F_y}{\partial y} + \frac{\partial F_z}{\partial z}$$

\textbf{Meaning:}
\begin{itemize}
\item $\nabla \cdot \vec{F} > 0$: source
\item $\nabla \cdot \vec{F} < 0$: sink
\item $\nabla \cdot \vec{F} = 0$: incompressible
\end{itemize}

\textbf{Curl:}
$$\nabla \times \vec{F} = \begin{vmatrix} \hat{i} & \hat{j} & \hat{k} \\ \frac{\partial}{\partial x} & \frac{\partial}{\partial y} & \frac{\partial}{\partial z} \\ F_x & F_y & F_z \end{vmatrix}$$

Measures rotation/vorticity of field
\end{formulabox}

\end{multicols}

\newpage

\begin{multicols}{3}

\begin{conceptbox}{PSE Conservation Laws}
\textbf{General Form:}
$$\frac{\partial \psi}{\partial t} + \nabla \cdot \vec{J}_\psi = S_\psi$$

where:
\begin{itemize}
\item $\psi$ = conserved quantity
\item $\vec{J}_\psi$ = flux
\item $S_\psi$ = source/sink
\end{itemize}

\textbf{Mass Conservation (Continuity):}
$$\frac{\partial \rho}{\partial t} + \nabla \cdot (\rho \vec{v}) = 0$$

\textbf{Heat Equation:}
$$\rho c_p \frac{\partial T}{\partial t} = \nabla \cdot (k \nabla T) + \dot{q}$$

\textbf{Species Diffusion:}
$$\frac{\partial C_A}{\partial t} = \nabla \cdot (D \nabla C_A) + r_A$$
\end{conceptbox}

\begin{formulabox}
\textbf{Transport Phenomena Unified:}

$$\text{Flux} = -(\text{Diffusivity}) \times \nabla(\text{Potential})$$

\begin{tabular}{llll}
\hline
Type & Flux & Diff. & Pot. \\
\hline
Heat & $\vec{q}$ & $k$ & $T$ \\
Mass & $\vec{N}_A$ & $D$ & $C_A$ \\
Mom. & $\vec{\tau}$ & $\mu$ & $\vec{v}$ \\
\hline
\end{tabular}

All follow same mathematical structure!
\end{formulabox}

\begin{conceptbox}{Line Integrals}
\textbf{Scalar Line Integral:}
$$\int_C f \, ds = \int_a^b f(\vec{r}(t)) |\vec{r}'(t)| dt$$

\textbf{Vector Line Integral (Work):}
$$\int_C \vec{F} \cdot d\vec{r} = \int_a^b \vec{F}(\vec{r}(t)) \cdot \vec{r}'(t) \, dt$$

\textbf{PSE:} Work done by pressure in pipe flow
\end{conceptbox}

\begin{conceptbox}{Surface Integrals}
\textbf{Flux Through Surface:}
$$\Phi = \iint_S \vec{F} \cdot \hat{n} \, dS$$

\textbf{PSE Applications:}
\begin{itemize}
\item Mass flux through membrane
\item Heat transfer through vessel wall
\item Momentum flux (drag force)
\end{itemize}
\end{conceptbox}

\begin{formulabox}
\textbf{Fundamental Theorems:}

\textbf{Divergence Theorem:}
$$\iiint_V (\nabla \cdot \vec{F}) \, dV = \iint_S \vec{F} \cdot \hat{n} \, dS$$

\textbf{Stokes' Theorem:}
$$\iint_S (\nabla \times \vec{F}) \cdot \hat{n} \, dS = \oint_C \vec{F} \cdot d\vec{r}$$

\textbf{Gradient Theorem:}
$$\int_C \nabla f \cdot d\vec{r} = f(\vec{r}_B) - f(\vec{r}_A)$$

Path-independent for conservative fields!
\end{formulabox}

\begin{notebox}
\textbf{Integration Strategy:}
\begin{enumerate}
\item Product? → Try integration by parts
\item Composition $f(g(x))$? → Try substitution
\item Rational function? → Partial fractions
\item Nothing works? → Numerical methods
\end{enumerate}
\end{notebox}

\columnbreak

\begin{conceptbox}{PSE Reactor Applications}
\textbf{Batch Reactor:}
$$\frac{dC_A}{dt} = r_A$$
$$t = C_{A0} \int_0^X \frac{dX}{-r_A/C_{A0}}$$

\textbf{CSTR:}
$$\tau = \frac{C_{A0} - C_A}{-r_A}$$

\textbf{PFR:}
$$\frac{dC_A}{dV} = \frac{r_A}{Q}$$
$$V = F_{A0} \int_0^X \frac{dX}{-r_A}$$
\end{conceptbox}

\begin{formulabox}
\textbf{Common Kinetic Forms:}

\textbf{First-order:}
$$-r_A = kC_A$$
$$C_A(t) = C_{A0}e^{-kt}$$

\textbf{Second-order:}
$$-r_A = kC_A^2$$
$$\frac{1}{C_A} - \frac{1}{C_{A0}} = kt$$

\textbf{Michaelis-Menten:}
$$-r_A = \frac{V_{max} C_A}{K_M + C_A}$$

\textbf{Arrhenius:}
$$k(T) = Ae^{-E_a/(RT)}$$
\end{formulabox}

\begin{conceptbox}{RTD (Residence Time Distribution)}
\textbf{Definition:}
$$\int_0^\infty E(t) \, dt = 1$$

\textbf{Mean Residence Time:}
$$\bar{t} = \int_0^\infty t E(t) \, dt$$

\textbf{Variance:}
$$\sigma^2 = \int_0^\infty t^2 E(t) \, dt - \bar{t}^2$$

Requires integration by parts!
\end{conceptbox}

\begin{notebox}
\textbf{Common Errors:}
\begin{itemize}
\item Confusing $\partial$ and $d$
\item Forgetting constant in partial derivatives
\item Wrong integration limits
\item Missing negative signs in flux laws
\item Dropping Jacobian factors
\item Not checking units!
\end{itemize}
\end{notebox}

\begin{formulabox}
\textbf{Quick Reference - Derivatives:}

\textbf{Exponential:}
$$\frac{d}{dx}[a^x] = a^x \ln a$$

\textbf{Logarithmic:}
$$\frac{d}{dx}[\log_a x] = \frac{1}{x \ln a}$$

\textbf{Hyperbolic:}
\begin{align*}
\frac{d}{dx}[\sinh x] &= \cosh x \\
\frac{d}{dx}[\cosh x] &= \sinh x \\
\frac{d}{dx}[\tanh x] &= \text{sech}^2 x
\end{align*}
\end{formulabox}

\columnbreak

\begin{conceptbox}{Optimization in PSE}
\textbf{Objective:} Maximize/minimize performance metric

\textbf{Examples:}
\begin{itemize}
\item Maximize selectivity $S = C_B/C_C$
\item Minimize residence time $\tau$
\item Maximize conversion $X$
\item Minimize energy consumption
\end{itemize}

\textbf{Constrained Optimization:}

Use Lagrange multipliers for constraints:
$$\mathcal{L} = f(x,y) + \lambda g(x,y)$$

Set $\nabla \mathcal{L} = 0$
\end{conceptbox}

\begin{formulabox}
\textbf{Numerical Differentiation:}

\textbf{Forward Difference:}
$$f'(x) \approx \frac{f(x+h) - f(x)}{h}$$

\textbf{Central Difference (more accurate):}
$$f'(x) \approx \frac{f(x+h) - f(x-h)}{2h}$$

\textbf{Second Derivative:}
$$f''(x) \approx \frac{f(x+h) - 2f(x) + f(x-h)}{h^2}$$
\end{formulabox}

\begin{conceptbox}{Heat Exchanger Analysis}
\textbf{Log-Mean Temperature Difference:}
$$\Delta T_{lm} = \frac{\Delta T_1 - \Delta T_2}{\ln(\Delta T_1/\Delta T_2)}$$

where:
\begin{itemize}
\item Counter-current: $\Delta T_1 = T_{h,in} - T_{c,out}$
\item $\Delta T_2 = T_{h,out} - T_{c,in}$
\end{itemize}

\textbf{Heat Transfer:}
$$Q = UA \Delta T_{lm}$$

Calculus needed for variable $U$ or non-linear $T$ profiles!
\end{conceptbox}

\begin{notebox}
\textbf{Physical Intuition Checks:}
\begin{itemize}
\item Concentration should decrease in batch reactor
\item Heat flows from hot to cold ($\nabla T < 0$ means $\vec{q} > 0$)
\item Conversion increases with time/volume
\item Flux has units (amount)/(area·time)
\item Always verify dimensional consistency!
\end{itemize}
\end{notebox}

\begin{formulabox}
\textbf{Special Integrals:}

$$\int xe^{ax} dx = \frac{e^{ax}}{a^2}(ax - 1) + C$$

$$\int \ln x \, dx = x\ln x - x + C$$

$$\int \frac{dx}{x^2 + a^2} = \frac{1}{a}\arctan\frac{x}{a} + C$$

$$\int \frac{dx}{\sqrt{a^2 - x^2}} = \arcsin\frac{x}{a} + C$$
\end{formulabox}

\begin{conceptbox}{Connection to Differential Equations}
Many PSE models lead to ODEs/PDEs:

\textbf{ODE:} $\frac{dy}{dx} = f(x,y)$

\textbf{PDE:} $\frac{\partial u}{\partial t} = \alpha \frac{\partial^2 u}{\partial x^2}$

Calculus is the foundation for solving these!

Next: ODEs and PDEs (covered later in roadmap)
\end{conceptbox}

\end{multicols}

\begin{center}
\rule{\textwidth}{0.4pt}

\textbf{Key Takeaway:} Calculus is the language of change and accumulation in PSE. \\
Master these tools to analyze reactors, separations, transport, and control systems!
\end{center}

\end{document}
